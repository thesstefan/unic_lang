\documentclass[11pt]{article}

\usepackage{amsthm}
\usepackage{amsmath}
\usepackage{amssymb}
\usepackage{enumerate}
\usepackage{geometry}[margins=1.5in]

\title{\textbf{Formal Languages \& Compiler Design \\ Homework 1}}
\author{Stefan Stefanache}

\begin{document}
    \maketitle
    \noindent \textbf{1.} Given the grammar 
    \begin{equation}
        G = \big(\{S, H\}, 
        \{b, c, d, e\}, 
        \{S \to b^2Se \hspace{0.25em} | \hspace{0.25em} H, H \to cHd^2 \hspace{0.25em} | \hspace{0.25em} cd\}, 
        S\big)
    \end{equation}
    find the language it generates.

    \vspace{1em}

    \begin{proof}
        Let us define the language
        \begin{equation}\label{lang:1}
            L = \{b^{2n}c^{m + 1}d^{2m + 1}e^{n} \hspace{0.25em}|\hspace{0.25em} n, m \in \mathbb{N}\}
            = \{L_{n, m} \hspace{0.25em}|\hspace{0.25em} n, m \in \mathbb{N}\}
        \end{equation} 
        and label $G$'s productions as follows:
        \begin{align}
            S &\to b^2Se \label{rule:1.1} \\
            S &\to H \label{rule:1.2} \\
            H &\to cHd^2 \label{rule:1.3} \\
            H &\to cd \label{rule:1.4}
        \end{align}
        Our goal is proving that $L = L(G)$ using the double inclusion technique. 
        \begin{enumerate}[\bfseries Step 1:]
            \item Show that $L \subseteq L(G)$:

            Using induction, one can easily show that for all $n \in \mathbb{N}$,
            \begin{align}
                H &\overset{n}{\underset{\eqref{rule:1.3}}{\implies}} c^{n}Hd^{2n} 
                \overset{1}{\underset{\eqref{rule:1.4}}{\implies}} c^{n + 1}d^{2n + 1} 
                \label{rule:1.5} \\
                S &\overset{n}{\underset{\eqref{rule:1.1}}{\implies}} b^{2n}Se^n 
                \overset{1}{\underset{\eqref{rule:1.2}}{\implies}} b^{2n}He^n
                \label{rule:1.6}
            \end{align}
            Therefore, we have that
            \[
                S \overset{*}{\underset{\eqref{rule:1.6}}{\implies}}
                b^{2k}He^k
                \overset{*}{\underset{\eqref{rule:1.5}}{\implies}}
                b^{2k}c^{p + 1}d^{2p + 1} e^k, \hspace{1em} \forall p, k \in \mathbb{N}
            \] 
            As a result, 
            \[
                S \overset{*}{\implies} w, \hspace{1em} \forall w \in L
            \] 
            so 
            \begin{equation}\label{incl:1.1}
                L \subseteq L(G)
            \end{equation}

            \item Show that $L \supseteq L(G)$:

            All elements in $L(G)$ are words/sequences derived from  $S$. Therefore,
            we have to show that all elements that can be derived from $S$ and don't contain
            any nonterminal symbols are contained by $L$. 

            We start by studying how the elements derived from $H$ look like. After applying 
            the \eqref{rule:1.3} rule for $m \in \mathbb{N}$ times, the result is of the form
            $c^mHd^{2m}$. To get rid of $H$, we apply \eqref{rule:1.4} one time to get
            a word/sequence of the form $c^{m + 1}d^{2m + 1}$. This is the only kind of 
            word/sequence that can be obtained starting from $H$.

            Now, we continue by looking at the possible elements derived from $S$. Analogously,
            we apply rule \eqref{rule:1.1} for $n \in \mathbb{N}$ times to get to the form
            $b^{2n}Se^n$. At this point, our only move is to transform the $S$ into $H$ using
            \eqref{rule:1.2}. Finally, we transform $H$ like previously discussed, to obtain a final
            word/sequence of the form $b^{2n}c^{m + 1}d^{2m + 1}e^n \in L$, with $n, m \in \mathbb{N}$.

            Since all possible words/sequences derived from $S$ can be found in $L$,
            we have that 
            \begin{equation}\label{incl:1.2}
                L \supseteq L(G)
            \end{equation}
        \end{enumerate}
        Finally, from \eqref{incl:1.1} and \eqref{incl:1.2}, we have that $L = L(G)$, so
        the language generated by the grammar $G$ is given by \eqref{lang:1}.
    \end{proof}

    \vspace{2em}

    \noindent \textbf{2.} Find grammars that generate the following languages:
    \begin{enumerate}[\hspace{2em} \bfseries A.]
        \item $L_1 = \{x^n y^n \hspace{0.25em} | \hspace{0.25em} n \in \mathbb{N}\}$, with proof
        \item $L_2 = \{a^n b^{2n} \hspace{0.25em} | \hspace{0.25em} n \in \mathbb{N}\}$, with proof
        \item $L_3 = \{a^n b^m \hspace{0.25em} | \hspace{0.25em} n, m \in \mathbb{N}^*\}$, 
            with proof using regular grammar
        \item $L_4 = \{x^{2n} \hspace{0.25em} | \hspace{0.25em} n \in \mathbb{N}\}, \hspace{0.5em}
            L_4' = \{x^{2n} \hspace{0.25em} | \hspace{0.25em} n \in \mathbb{N}^*\}$,
            with proof using regular grammar
        \item $\mathbb{N}$
        \item All arithmetic expressions containing $a$ as operand, $+$, $*$ as operators
            and $()$.
    \end{enumerate}

    \vspace{1em}

    \begin{proof}
    For all proofs, we'll be using a similar method as in the previous exercise: find a grammar
    $G_1$ and then prove that $L_1 = L(G_1)$ using the double inclusion technique.
    \begin{enumerate}[\hspace{2em} \bfseries A.]
        \item Let us define the grammar 
            \begin{equation}\label{grammar:2.A}
                G_1 = \big(\{A\}, \{x, y\}, \{A \to xAy \hspace{0.25em}|\hspace{0.25em} \epsilon\}, A\big)
            \end{equation}
        and label its rules as
        \begin{align}
            A &\to xAy \label{rule:2A.1} \\
            A &\to \epsilon \label{rule:2A.2}
        \end{align}
        \begin{enumerate}[\bfseries Step 1.]
            \item Show that $L_1 \subseteq L(G_1)$:

            Using induction, one can easily show that 
            $A \overset{n}{\underset{\eqref{rule:2A.1}}{\implies}} x^nAy^n$. 
            
            Since $x^nAy^n \overset{1}{\underset{\eqref{rule:2A.2}}{\implies}} x^ny^n$,
            we have that
            \begin{equation}
                A \overset{*}{\implies} x^ny^n \in L_1
            \end{equation}
            Therefore, $A \overset{*}{\implies} w, \hspace{0.25em} \forall w \in L_1$, so
            \begin{equation}\label{incl:2A.1}
                L_1 \subseteq L(G_1)
            \end{equation}

            \vspace{1em}

            \item Show that $L_1 \supseteq L(G_1)$:

            We have to prove that each element derived from $A$ that doesn't contain
            any nonterminal symbols is contained by $L_1$. Starting from $A$, we apply
            rule $\eqref{rule:2A.1}$ for $n \in \mathbb{N}$ times and reach a result
            of the form $x^nAy^n$. To get rid of the nonterminal symbol $A$, we use
            rule \eqref{rule:2A.2} to obtain the word/sequence $x^ny^n$.

            Since this is the only possible way of sequencing transformations (note that
            this also works for $n=0$), all elements derived from $A$ with no
            nonterminal terms are of the form $x^ny^n \in L_1$. Therefore, 
            \begin{equation}\label{incl:2A.2}
                L_1 \supseteq L(G_1) 
            \end{equation}
        \end{enumerate}
        Using $\eqref{incl:2A.1}$ and $\eqref{incl:2A.2}$, we have that $L_1 = L(G_1)$, so
        we've proved that the language $L_1$ is generated by the grammar given by \eqref{grammar:2.A}.

        \vspace{1em}

        \item Let us define the grammar 
            \begin{equation}\label{grammar:2.B}
                G_2 = \big(\{B\}, \{a, b\}, \{B \to aBb^2 \hspace{0.25em}|\hspace{0.25em} ab^2\}, B\big)
            \end{equation}
        and label its rules as
        \begin{align}
            B &\to aBb^2 \label{rule:2B.1} \\
            B &\to ab^2 \label{rule:2B.2}
        \end{align}
        \begin{enumerate}[\bfseries Step 1.]
            \item Show that $L_2 \subseteq L(G_2)$:

            Let $n \in \mathbb{N}^*$. Using induction, one can easily show that 
            \[
                B \overset{n - 1}{\underset{\eqref{rule:2B.1}}{\implies}} a^{n - 1}Bb^{2n - 2}
            \] 
            Since $a^{n - 1}Bb^{2n - 2} \overset{1}{\underset{\eqref{rule:2B.2}}{\implies}} a^nb^{2n}$,
            we have that
            \begin{equation}
                B \overset{*}{\implies} a^nb^{2n} \in L_2
            \end{equation}
            Therefore, $B \overset{*}{\implies} w, \hspace{0.25em} \forall w \in L_2$, so
            \begin{equation}\label{incl:2B.1}
                L_2 \subseteq L(G_2)
            \end{equation}

            \vspace{1em}

            \item Show that $L_2 \supseteq L(G_2)$:

            We have to prove that each element derived from $B$ that doesn't contain
            any nonterminal symbols is contained by $L_2$. Starting from $B$, we apply
            rule $\eqref{rule:2B.1}$ for $n - 1 \in \mathbb{N}$ times and reach a result
            of the form $a^{n - 1}Bb^{2n - 2}$. To get rid of the nonterminal symbol $B$, we use
            rule \eqref{rule:2B.2} to obtain the word/sequence $a^nb^{2n}$.

            Since this is the only possible way of sequencing transformations (note that
            this also works for $n=0$), all elements derived from $B$ with no
            nonterminal terms are of the form $a^nb^{2n} \in L_2$. Therefore, 
            \begin{equation}\label{incl:2B.2}
            L_2 \supseteq L(G_2) 
            \end{equation}
        \end{enumerate}
        Using $\eqref{incl:2B.1}$ and $\eqref{incl:2B.2}$, we have that $L_2 = L(G_2)$, so
        we've proved that the language $L_2$ is generated by the grammar given by \eqref{grammar:2.B}.

        \vspace{1em}

        \item Let us define the \textbf{regular} grammar 
            \begin{equation}\label{grammar:2.C}
                G_3 = \big(\{A, B\}, \{a, b\}, 
                \{A \to aA \hspace{0.25em}|\hspace{0.25em} aB, 
                B \to bB \hspace{0.25em}|\hspace{0.25em} b\}, A\big)
            \end{equation}
        and label its rules as
        \begin{align}
            A &\to aA \label{rule:2C.1} \\
            A &\to aB \label{rule:2C.2} \\
            B &\to bB \label{rule:2C.3} \\
            B &\to b \label{rule:2C.4}
        \end{align}
        \begin{enumerate}[\bfseries Step 1.]
            \item Show that $L_3 \subseteq L(G_3)$:

            Let $n, m \in \mathbb{N}^*$. Using induction twice, one can easily show that 
            \[
                A \overset{n - 1}{\underset{\eqref{rule:2C.1}}{\implies}} a^{n - 1}A
                \overset{1}{\underset{\eqref{rule:2C.2}}{\implies}} a^nB
                \overset{m - 1}{\underset{\eqref{rule:2C.3}}{\implies}} a^nb^{m - 1}B
                \overset{1}{\underset{\eqref{rule:2C.4}}{\implies}} a^nb^m
            \] 
            Therefore, we have that
            \begin{equation}
                A \overset{*}{\implies} a^nb^m \in L_3
            \end{equation}
            Therefore, $A \overset{*}{\implies} w, \hspace{0.25em} \forall w \in L_3$, so
            \begin{equation}\label{incl:2C.1}
                L_3 \subseteq L(G_3)
            \end{equation}

            \vspace{1em}

            \item Show that $L_3 \supseteq L(G_3)$:

            We have to prove that each element derived from $A$ that doesn't contain
            any nonterminal symbols is contained by $L_3$. Starting from $A$, we apply
            rule \eqref{rule:2C.1} for $n - 1 \in \mathbb{N}$ times and reach a result
            of the form $a^{n-1}A$. To get rid of the nonterminal symbol $A$, we use
            rule \eqref{rule:2B.2} to obtain the new form $a^nB$.

            Now, we can apply rule \eqref{rule:2C.3} for $m - 1 \in \mathbb{N}$ times
            to reach the form  $a^nb^{m - 1}B$. Getting rid of the nonterminal $B$ 
            is done by using the rule \eqref{rule:2C.4} and reaching the final form
            $a^nb^m \in L_3$.

            Since this is the only possible way of sequencing transformations, 
            all elements derived from $A$ with no nonterminal terms are of the 
            form $a^nb^{m} \in L_3$. Therefore, 
            \begin{equation}\label{incl:2C.2}
                L_3 \supseteq L(G_3) 
            \end{equation}
        \end{enumerate}
        Using $\eqref{incl:2C.1}$ and $\eqref{incl:2C.2}$, we have that $L_3 = L(G_3)$, so
        we've proved that the language $L_3$ is generated by the grammar given by \eqref{grammar:2.C}.

        \item Let us define the \textbf{regular} grammar 
            \begin{equation}\label{grammar:2.D}
                G_4 = \big(\{S, A, B\}, \{x\}, 
                \{S \to xA \hspace{0.25em}|\hspace{0.25em} \epsilon, 
                A \to xB \hspace{0.25em}|\hspace{0.25em} x,
                B \to xA\}, S\big)
            \end{equation}
        and label its rules as
        \begin{align}
            S &\to xA \label{rule:2D.1} \\
            S &\to \epsilon \label{rule:2D.2} \\
            A &\to xB \label{rule:2D.3} \\
            A &\to x \label{rule:2D.4} \\
            B &\to xA \label{rule:2D.5}
        \end{align}
        One trick we're using here is making sure that $A$ is always accompanied
        by an odd power of $x$. That way, we either use \eqref{rule:2D.4} to transform
        into something of the form $x^{2k}$, or transform it into $xB$ using 
        \eqref{rule:2D.3} and cycle back with \eqref{rule:2D.5}, getting something of
        the form $x^{2k + 1}A$ again. By doing it this way, we make sure that
        only odd powers are covered.
        \begin{enumerate}[\bfseries Step 1.]
            \item Show that $L_4 \subseteq L(G_4)$:

            To enable the cycle trick discussed above,
            we'll prove by induction that 
            \begin{equation}\label{cycle}
                A \overset{*}{\implies} x^{2n + 1}, \hspace{0.25em} \forall n \in \mathbb{N}
            \end{equation}
            The base case for $n=0$ holds since $A \overset{1}{\underset{\eqref{rule:2D.1}}{\implies}} x$.
            Now, let's take $k \in \mathbb{N}$ and assume that
            \begin{equation}\label{ind:2D.1}
                A \overset{*}{\implies} x^{2k + 1}
            \end{equation}
            Then,
            \[
                A \overset{1}{\underset{\eqref{rule:2D.3}}{\implies}} xB
                \overset{1}{\underset{\eqref{rule:2D.5}}{\implies}} x^2A
                \overset{*}{\underset{\eqref{ind:2D.1}}{\implies}} x^{2k + 3}
            \] 
            This proves that $P(k) \implies P(k + 1)$ is true, where 
             \[
                 P(n): A \overset{*}{\implies} x^{2n + 1}
            \] 
            Therefore, since both $P(0)$ and $P(k) \implies P(k + 1)$ hold we've proved 
            that \eqref{cycle} is true. Now, we have that
            \[
                S \overset{1}{\underset{\eqref{rule:2D.1}}{\implies}} xA
                \overset{*}{\underset{\eqref{cycle}}{\implies}} x^{2n} \in L_4
            \] 
            Therefore, $S \overset{*}{\implies} w, \hspace{0.25em} \forall w \in L_4$, so
            \begin{equation}\label{incl:2D.1}
                L_4 \subseteq L(G_4)
            \end{equation}

            \vspace{1em}

            \item Show that $L_4 \supseteq L(G_4)$:

            We have to prove that each element derived from $S$ that doesn't contain
            any nonterminal symbols is contained by $L_4$. Starting from $S$, we 
            apply rule \eqref{rule:2D.1} to obtain $xA$. Now, we sequentially apply
            the rules \eqref{rule:2D.3} and \eqref{rule:2D.5} for $k \in \mathbb{N}$ 
            times to obtain something of the form $x^{2k + 1}A$. Now, we can
            apply \eqref{rule:2D.4} to get rid of the nonterminal symbol $A$ and
            get the final sequence of the form $x^{2k} \in L_4$.

            We can also start from $S$ and apply \eqref{rule:2D.2} to obtain
            $\epsilon = x^0 \in L_4$.

            Since these are the only possible way of sequencing transformations, 
            all elements derived from $A$ with no nonterminal terms are of the 
            form $a^nb^{m} \in L_3$. Therefore, 
            \begin{equation}\label{incl:2D.2}
                L_4 \supseteq L(G_4) 
            \end{equation}
        \end{enumerate}
        Using $\eqref{incl:2D.1}$ and $\eqref{incl:2D.2}$, we have that $L_4 = L(G_4)$, so
        we've proved that the language $L_4$ is generated by the grammar given by \eqref{grammar:2.D}.

        We do exactly the same thing for $L_4'$, but by removing the \eqref{rule:2D.2} rule
        from the production set of the grammar, since we're now dealing with only
        positive powers of $x$. One can define the adapted \textbf{regular} grammar
        \begin{equation}
            G_4' = \big(\{S, A, B\}, \{x\}, 
            \{S \to xA, 
            A \to xB \hspace{0.25em}|\hspace{0.25em} x,
            B \to xA\}, S\big)
        \end{equation}
        and follow the same steps as above to prove that $L_4' = L(G_4')$.

        \vspace{1em}

        \item Let us define the grammar
        \begin{equation}
            G_5 = \big(\{S, A, B, C, D\}, \Sigma, P, S\big)
        \end{equation}
        where the nonterminal symbols are the 10 digits,
        \begin{equation}
            \Sigma = \{0, 1, 2, 3, 4, 5, 6, 7, 8, 9\}
        \end{equation}
        and the production set is given by
        \begin{equation}
            P = \{
                S \to 0 \hspace{0.25em}|\hspace{0.25em} A,
                A \to B \hspace{0.25em}|\hspace{0.25em} BC,
                C \to D \hspace{0.25em}|\hspace{0.25em} DC,
                D \to 0 \hspace{0.25em}|\hspace{0.25em} B,
                B \to 1 \hspace{0.25em} | \hspace{0.25em} 2
                \hspace{0.25em} | \hspace{0.25em} 3 \hspace{0.25em} |
                \ldots | \hspace{0.25em} 9
            \}
        \end{equation}
        This grammar generates the set of natural numbers $\mathbb{N}$ ($L(G_5) = \mathbb{N}$) 
        and its rules can be described as follows:
        $B$ denotes a non-zero digit, $D$ denotes any digit, $C$ denotes a
        sequence of digits, and $A$ denotes a natural positive number. $S$ 
        can be transformed to a positive number ($A$) or 0.

        \item Let us define the grammar
        \begin{equation}
            G_6 = \big(\{S, A, b\}, \{a, +, *, (, )\}, P, S\big)
        \end{equation}
        Using the production set
        \begin{equation}
            P = \{
                S \to S + a \hspace{0.25em}|\hspace{0.25em} A,
                A \to A * a \hspace{0.25em}|\hspace{0.25em} B,
                B \to (S) \hspace{0.25em}|\hspace{0.25em} a
            \}
        \end{equation}
        we have that $L(G_6)$ covers all arithmetic expressions containing 
        $a$ as an operand, +, * as operators and (). Compound statements
        can be enabled by the cyclical nature of the rules $S \to A \to B \to S \to \ldots$.
        The first rule enables sums, the second products, and the third compounds
        statements.
    \end{enumerate}
    \end{proof}
\end{document}
